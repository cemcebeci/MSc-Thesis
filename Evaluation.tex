\section{Goals}

\section{Conditions}

\section{Baseline}
In this section, we describe the baseline we use to evaluate the efficiency of fuzzers generated by RLC.
As a baseline, we need a simpler method of automatically generating fuzzers for game descriptions. Any such method
needs to establish an abstraction for how a game is described. We have chosen to use the abstraction of OpenSpiel, 
Google DeepMind's framework for applying reinforcement learning methods to games (TODO: I probably need to describe
 OpenSpiel in greater detail.).

The OpenSpiel repository includes some example games. We describe a simple method of generating fuzz targets for these games.
These games, modified minimally to introduce bugs the fuzzer can find, form the baseline of our evaluation.

\subsubsection{Generating fuzzers for OpenSpiel games}
(TODO: I adapted this from the OpenSpiel paper, should figure out how to cite that.)
Games in OpenSpiel are described as producedural extensive-form games. Where a game has:
\begin{itemize}
    \item A finite set of players. Including a special player representing chance.
    \item A finite set of all possible actions players can take in all game states.
    \item A finite set of histories. Each history is a sequence of actions that were taken from the start
    of the game.
    \item A finite set of terminal histories that represent a finished game.
    \item A utility for each player for each terminal history.
    \item A player assigned to take the next action for each non-terminal history. Including a special player representing simultaneous states, 
        where players act simultaneously choosing a joint action.
    \item A set of states, where each state is a set of histories such that histories in the same state can not be distinguished by the acting player.
\end{itemize}
Complete game descriptions are objects that implement some interface methods exposing the elements 
described above, along with some utility methods to simplify moving from a history to its successors.
In particular, OpenSpiel game descriptions implement the following methods that are useful for generating a fuzzer:
\begin{itemize}
    \item \texttt{Game::NewInitialState}
    \item \texttt{State::isTerminal}
    \item \texttt{State::isChanceNode}
    \item \texttt{State::isSimultaneousNode}
    \item \texttt{State::LegalChanceOutcomes}
    \item \texttt{State::LegalActions}
    \item \texttt{State::ApplyAction}
\end{itemize}
Depending only on these functions, we can generate a fuzz target analogous to the ones we generate for RL descriptions 
as shown in Algorithm~\ref{alg:var}, where $pickOne$ is a function that picks an element of the given set consuming the next $log_2(n)$ bits of 
the fuzz input. After generating the fuzz target, we plug the fuzz target into LLVM's libFuzzer
to generate the complete fuzzer.

\begin{algorithm}[H]
    \caption{Fuzzing and OpenSpiel game}
    \label{alg:var}
    \begin{algorithmic}[1]
    \STATE $state \gets game.NewInitialState()$
    \WHILE{$state.isTerminal()$}
        \IF{$state.isChanceNode()$}
            \STATE $nextAction \gets \pmb{pickOne}(state.LegalChanceOutcomes())$
        \ELSIF{$state.isSimultaneousNode()$}
            \STATE $actions \gets []$
            \FOR {$player \in players$}
                \STATE $actions.append(\pmb{pickOne}(state.LegalActions(player)))$
            \ENDFOR
            \STATE $nextAction \gets ApplyAction(actions)$
        \ELSE
            \STATE $nextAction \gets \pmb{pickOne}(state.LegalActions())$
        \ENDIF
        \STATE $state.ApplyAction(nextAction)$
    \ENDWHILE
    \end{algorithmic}
\end{algorithm}

It should be noted that OpenSpiel games may have two kinds of chance nodes. Explicit stochastic chance nodes
expose multiple legal actions, as well as a probability distribution over those actions. On the other hand, 
sampled stochastic chance nodes expose a single action with non-deterministic behaviour. This fuzz target is 
only suitable for games with no samples stochastic chance nodes since the fuzz target has to be deterministic
with respect to the fuzz input.


\subsubsection{Tic tac toe}

\section{Results}