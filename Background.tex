In this chapter, we introduce the motivation behind RL and provide a brief overview of the language.
Then, we describe the language features tightly linked with fuzzing in greater detail, aiming to supply the reader enough of a background to 
be able to comprehend how our fuzzer generation technique works.

\section{What is RL}
RL (RuleBook) is a domain specific language to describe multi-agent interactions, most commonly games \cite{RLC}.
RLC is the compiler for RL. RL is created to solve a problem with the way games are implemented today that makes it difficult to apply automated analysis and machine learning techniques to games.
Today, games are developed as a collection of event handlers.
The developer specifies the actions players can take, as well as the effect of those actions on the game state.
The actions are described in isolation, one has to analyze which actions can follow which others in order to simulate a game trace.
In contrast, when we describe games to other humans in natural language, as RL does, we describe them as sequential scenarios where the players get to take actions as part of the scenario.
We expect the actions to be taken at certain points in the scenario, with certain constraints.
As an example, compare two natural language descriptions of a single hand of blackjack to illustrate the difference.

A description in terms of events, their results, and their constraints could be:
\begin{itemize}
    \item Players can take additional cards.
    \item Players can pass.
    \item It is always a single player’s turn.
    \item When a player takes an additional card, they get a card from the top of the deck.
    \item A player may not take an additional card unless it is their turn.
    \item A player may not take an additional card if they have passed before.
    \item When a player takes an additional card or passes, it becomes the next player’s turn.
    \item Players start with 2 cards in their hands.
    \item If all players have passed once, the player with the hand closest to 21 wins.
\end{itemize}

On the other hand, a description as a sequential scenario could be:
\begin{itemize}
    \item The game starts by dealing each player a hand of 2 cards.
    \item Until all players have passed once, each player that has not previously passed decides whether they want to take an additional card or pass.
    \item The players with the hand closest to 21 wins.
\end{itemize}
The description organized into event handlers makes games harder to comprehend and potentially harder to develop for humans.
However, RL's main focus is not really on human readability.
It tries to solve the problem that modern games implemented as event handlers are very difficult to perform automated analyses on.
A sequential description makes game descriptions more amenable to automated analysis techniques.
These techniques can range from fuzzing the game with random actions hoping to discover invalid game states early in development, to training ML models on the game.

\section{Description of the RL language}
RL is an imperative programming language and its structure is similar to other common imperative
languages such as Python or C. An RL program is composed of functions, and the entry point is a special
function with signature \texttt{fun main() -> Int}. Each function has a list of statements to be executed sequentially.
These statements include what one might expect to find in an imperative language, such as 
variable declarations, assignments, if statements, while loops, and function calls. RL extends this 
familiar language structure with a couple key features.

\subsection{Actions}
In addition to regular functions, RL features another function-like construct called actions.
The main difference between a function and an action is in their control flow semantics.
When a function is called, it runs until it executes a return statement.
Then, all variables local to the function body are discarded and solely the return value is returned to the caller.
In contrast, when an action is invoked it runs until it executes a special kind of statement which causes it to suspend.
Then, it returns an action object that describes the current state of execution, including the values of local variables as well as where the execution has suspended.
The caller can invoke methods of this returned object to resume the execution from its previous state.
Action bodies do not contain explicit return statements, actions are meant to model processes, not values.

Two kinds of statements suspend an action's executions: Action statements and actions statements
An action statement has the syntax \texttt{act action\_name( param\_type param\_name, ...)}.
When the action suspends on an action statement, it can be resumed by calling the method of the action object that shares a signature with the action statement.
The action then resumes execution starting from the statement immediately succeeding the action statement.
In addition, the action statement's parameters are accessible by successive statements.
The parameters assume the value of arguments passed to the function that resumes the execution.

An actions statement has the following syntax
\begin{lstlisting}
actions:
    action_statement
    non_action_statement
    non_action_statement
    ...

    action_statement
    non_action_statement
    non_action_statement
    ...
\end{lstlisting}
An action suspended on an actions statement can be resumed on any of the action statements contained in it.
When resumed, only statements until the next action statement are executed before proceeding to the actions statement's successor.


Not all variables in an action's body are exposed to callers.
Variables declared with the keyword \texttt{frm} are stored in the action's frame.
They persist across suspensions and are reachable from outside the action.
On the other hand, variables declared with the keyword \texttt{let} are temporary values.
They are discarded when the action's execution suspends, just like local variables in regular functions.
As an example, consider the following program:

\lstinputlisting{action_example.rl}

The main function invokes \texttt{nim(14)}.
The action initializes the variables \texttt{current\_player} and \texttt{remaining\_sticks}, enters the loop and suspends at line 7.
When it suspends, the action returns a \texttt{Nim} object, which holds the variables
 \texttt{current\_player} and \texttt{remaining\_sticks}, exposes the functions \texttt{pick\_up\_sticks(Int count)} and \texttt{is\_done\(\)} 
 and "remembers" that the action is suspended at the \texttt{pick\_up\_sticks} action statement.
The main function then stores this object in the variable \texttt{game}, and calls the method it exposes multiple times.
Each of these calls but the last one execute one iteration of the while loop before suspending on line 7 once again.
During the last \texttt{pick\_up\_sticks} call at line 18, the action's execution exits the while loop and terminates, since there are no other action statements in the action's body.
After this point, \texttt{game.is\_done()} returns \texttt{True} and calling \texttt{pick\_up\_sticks} again will result in a crash since the action is not suspended on that statement.
The action's frame variables are still accessible through the \texttt{Nim} object.

\subsection{Preconditions}
In RL, every function, action and subaction statement can have a list of preconditions attached to it.
Any expression that evaluates to a boolean value can be a precondition, and preconditions can use the parameters of the function, action or action statement they are associated with.
Unless optimizations are turned on while compiling, a function call or action instantiation with arguments that violate its target's preconditions results in a crash.

Being able to express preconditions is crucial for describing simulations.
For instance, consider the nim example. The action 
\begin{lstlisting}
act nim(Int num_sticks) -> Nim:
    frm winner : Int
    frm current_player = 0
    frm remaining_sticks = num_sticks

    while remaining_sticks > 0:
        act pick_up_sticks(Int count)
        remaining_sticks = remaining_sticks - count
        current_player = 1 - current_player

    winner = current_player
\end{lstlisting}
allows players to pick up more sticks than there are in the game, or even a negative number of sticks.
Moreover, the game is not restricted to start with a positive number of sticks in the first place.
Enhancing the description with preconditions increases the description's readability, eases debugging and boosts the potential of automated analysis methods.
\begin{lstlisting}
act nim(Int num_sticks) {num_sticks > 0} -> Nim:
    frm winner : Int
    frm current_player = 0
    frm remaining_sticks = num_sticks

    while remaining_sticks > 0:
        act pick_up_sticks(Int count) {
            count > 0,
            count <= 4,
            count <= remaining_sticks
        }
        remaining_sticks = remaining_sticks - count
        current_player = 1 - current_player

    winner = current_player
\end{lstlisting}
To summarize, RL is a domain-specific language aimed at writing, reading and automatically analyzing game descriptions easier, as well as running machine learning methods on them.
It has a Python-like syntax. RL allows describing games as sequential scenarios mainly through actions, which are stateful suspendable procedures.
In addition, RL supports specifying preconditions for functions, actions and subactions, further facilitating the application of analysis techniques.