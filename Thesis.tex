% A LaTeX template for MSc Thesis submissions to 
% Politecnico di Milano (PoliMi) - School of Industrial and Information Engineering
%
% S. Bonetti, A. Gruttadauria, G. Mescolini, A. Zingaro
% e-mail: template-tesi-ingind@polimi.it
%
% Last Revision: October 2021
%
% Copyright 2021 Politecnico di Milano, Italy. NC-BY

\documentclass{Configuration_Files/PoliMi3i_thesis}

%------------------------------------------------------------------------------
%	REQUIRED PACKAGES AND  CONFIGURATIONS
%------------------------------------------------------------------------------

% CONFIGURATIONS
\usepackage{parskip} % For paragraph layout
\usepackage{setspace} % For using single or double spacing
\usepackage{emptypage} % To insert empty pages
\usepackage{multicol} % To write in multiple columns (executive summary)
\setlength\columnsep{15pt} % Column separation in executive summary
\setlength\parindent{10pt} % Indentation
\raggedbottom  

% PACKAGES FOR TITLES
\usepackage{titlesec}
% \titlespacing{\section}{left spacing}{before spacing}{after spacing}
\titlespacing{\section}{0pt}{3.3ex}{2ex}
\titlespacing{\subsection}{0pt}{3.3ex}{1.65ex}
\titlespacing{\subsubsection}{0pt}{3.3ex}{1ex}
\usepackage{color}

% PACKAGES FOR LANGUAGE AND FONT
\usepackage[english]{babel} % The document is in English  
\usepackage[utf8]{inputenc} % UTF8 encoding
\usepackage[T1]{fontenc} % Font encoding
\usepackage[11pt]{moresize} % Big fonts

% PACKAGES FOR IMAGES
\usepackage{graphicx}
\usepackage{svg}
\usepackage{transparent} % Enables transparent images
\usepackage{eso-pic} % For the background picture on the title page
%\usepackage{subfig} % Numbered and caption subfigures using \subfloat.
\usepackage[export]{adjustbox}
\usepackage{tikz} % A package for high-quality hand-made figures.
\usetikzlibrary{}
\graphicspath{{./Images/}} % Directory of the images
\usepackage{caption} % Coloured captions
\usepackage{subcaption}
\usepackage{xcolor} % Coloured captions
\usepackage{amsthm,thmtools,xcolor} % Coloured "Theorem"
\usepackage{float}

% STANDARD MATH PACKAGES
\usepackage{amsmath}
\usepackage{amsthm}
\usepackage{amssymb}
\usepackage{amsfonts}
\usepackage{bm}
\usepackage{cancel}
\usepackage[overload]{empheq} % For braced-style systems of equations.
\usepackage{fix-cm} % To override original LaTeX restrictions on sizes

% PACKAGES FOR TABLES
\usepackage{tabularx}
\usepackage{longtable} % Tables that can span several pages
\usepackage{colortbl}

% PACKAGES FOR ALGORITHMS (PSEUDO-CODE)
\usepackage{algorithm}
\usepackage{algorithmic}

% PACKAGES FOR REFERENCES & BIBLIOGRAPHY
\usepackage[colorlinks=true,linkcolor=black,anchorcolor=black,citecolor=black,filecolor=black,menucolor=black,runcolor=black,urlcolor=black]{hyperref} % Adds clickable links at references
\usepackage{cleveref}
\usepackage[square, numbers, sort&compress]{natbib} % Square brackets, citing references with numbers, citations sorted by appearance in the text and compressed
\bibliographystyle{abbrvnat} % You may use a different style adapted to your field

% OTHER PACKAGES
\usepackage{pdfpages} % To include a pdf file
\usepackage{afterpage}
\usepackage{lipsum} % DUMMY PACKAGE
\usepackage{fancyhdr} % For the headers
\usepackage{fancyvrb}
\usepackage[acronym]{glossaries}
\usepackage{enumitem} 
\fancyhf{}

\usepackage{listings}

\usepackage{xcolor}

\definecolor{codegreen}{rgb}{0,0.6,0}
\definecolor{codegray}{rgb}{0.5,0.5,0.5}
\definecolor{codepurple}{rgb}{0.58,0,0.82}
\definecolor{backcolour}{rgb}{0.95,0.95,0.92}

\lstdefinestyle{mystyle}{
    backgroundcolor=\color{backcolour},   
    commentstyle=\color{codegreen},
    keywordstyle=\color{magenta},
    numberstyle=\tiny\color{codegray},
    stringstyle=\color{codepurple},
    basicstyle=\ttfamily\footnotesize,
    breakatwhitespace=false,         
    breaklines=true,                 
    captionpos=b,                    
    keepspaces=true,                 
    numbers=left,                    
    numbersep=5pt,                  
    showspaces=false,                
    showstringspaces=false,
    showtabs=false,                  
    tabsize=2
}

\lstset{style=mystyle}

% Input of configuration file. Do not change config.tex file unless you really know what you are doing. 
\input{Configuration_Files/config}

%----------------------------------------------------------------------------
%	NEW COMMANDS DEFINED
%----------------------------------------------------------------------------

% EXAMPLES OF NEW COMMANDS
\newcommand{\bea}{\begin{eqnarray}} % Shortcut for equation arrays
\newcommand{\eea}{\end{eqnarray}}
\newcommand{\e}[1]{\times 10^{#1}}  % Powers of 10 notation

%----------------------------------------------------------------------------
%	ADD YOUR PACKAGES (be careful of package interaction)
%----------------------------------------------------------------------------

%----------------------------------------------------------------------------
%	ADD YOUR DEFINITIONS AND COMMANDS (be careful of existing commands)
%----------------------------------------------------------------------------

\input{Thesis_Acronyms}

%----------------------------------------------------------------------------
%	BEGIN OF YOUR DOCUMENT
%----------------------------------------------------------------------------

\begin{document}

\fancypagestyle{plain}{%
\fancyhf{} % Clear all header and footer fields
\fancyhead[RO,RE]{\thepage} %RO=right odd, RE=right even
\renewcommand{\headrulewidth}{0pt}
\renewcommand{\footrulewidth}{0pt}}

%----------------------------------------------------------------------------
%	TITLE PAGE
%----------------------------------------------------------------------------

\pagestyle{empty} % No page numbers
\frontmatter % Use roman page numbering style (i, ii, iii, iv...) for the preamble pages

\puttitle{
	title=Automated generation of efficient fuzzers for games using libFuzzer and RuleBook, % Title of the thesis
	name=Cem Cebeci, % Author Name and Surname
	course=Computer Science and Engineering \\Ingegneria Informatica, % Study Programme (in Italian)
	ID  =  10837160,  % Student ID number (numero di matricola)
	advisor= Prof. Giovanni Agosta, % Supervisor name
	coadvisor= Massimo Fioravanti, % Co-Supervisor name, remove this line if there is none
	academicyear={2023-2024},  % Academic Year
} % These info will be put into your Title page 

%----------------------------------------------------------------------------
%	PREAMBLE PAGES: ABSTRACT (inglese e italiano), EXECUTIVE SUMMARY
%----------------------------------------------------------------------------
\pagebreak

\startpreamble
\setcounter{page}{1} % Set page counter to 1

% ABSTRACT IN ENGLISH
\chapter*{Abstract} 
Fuzzing is the practice of repeatedly calling a program with semi-random inputs.
It is a popular technique to discover bugs and vulnerabilities in software.
However, fuzzing is not very practical when applied to input intensive programs such as games.
A major factor in this is that games often have complex input patterns.
They do not operate on a single large input but instead on a multitude of actions taken on them by the players.
These actions must be taken according to some potentially complex rules, a particular action is only available on some game states and only with certain parameters.
Therefore, even if a fuzzer knows all possible actions it can take on a game, generating valid sequences of actions can be a difficult task.
An invalid sequence of actions does not make an interesting fuzz input, since the game is expected to fail with such an input.
Consequently, frequently generating invalid sequences of actions decreases the fuzzer performance significantly.
Luckily, information about when an action can be taken and the constraints on its parameters should be available in the game logic, since the game needs to validate the actions players take and display error messages when invalid actions are taken.
Therefore, compiler techniques should be able to extract and exploit this information already present in the game rules to produce high-quality fuzzers.

In this work, we propose a method to generate fuzzers for games that utilize information about the dynamic availability of actions.
We describe a technique to parse an arbitrary sequence of bytes as a sequence of actions on the game, then detail how we minimize the number of invalid actions in the parsed action sequences.
We implement this technique in the RuleBook language compiler. RuleBook is a DSL for describing games and the RuleBook compiler has the information relevant to our technique available at compile time.
We measure the performance of four versions of fuzzers generated with this method, varying on how much information about the target game they utilize, along with two versions of fuzzers generated using OpenSpiel's game implementations.
Using three sample games, we demonstrate that the fuzzers generated from RL descriptions are as performant as their OpenSpiel counterparts.
Furthermore, we show that between RL fuzzers, versions that make the most use of the information RL exposes perform the best.
\\
\\
\textbf{Keywords:} compilers, fuzzing, games

% ABSTRACT IN ITALIAN
\chapter*{Abstract in lingua italiana}
Il fuzzing è la pratica di richiamare ripetutamente un programma con input semi-casuali.
È una tecnica popolare per scoprire bug e vulnerabilità nel software.
Tuttavia, il fuzzing non è molto pratico se applicato a programmi ad alta intensità di input come i giochi.
Un fattore importante è che i giochi hanno spesso modelli di input complessi.
Non operano su un singolo input di grandi dimensioni, ma su una moltitudine di azioni eseguite dai giocatori.
Queste azioni devono essere eseguite secondo regole potenzialmente complesse; una particolare azione è disponibile solo in alcuni stati del gioco e solo con determinati parametri.
Pertanto, anche se un fuzzer conosce tutte le possibili azioni che può compiere su un gioco, generare sequenze di azioni valide può essere un compito difficile.
Una sequenza di azioni non valida non costituisce un input interessante per il fuzz, poiché ci si aspetta che il gioco fallisca con tale input.
Di conseguenza, la generazione frequente di sequenze di azioni non valide riduce notevolmente le prestazioni del fuzzer.
Fortunatamente, le informazioni su quando un'azione può essere eseguita e i vincoli sui suoi parametri dovrebbero essere disponibili nella logica di gioco, poiché il gioco deve convalidare le azioni eseguite dai giocatori e visualizzare messaggi di errore quando vengono eseguite azioni non valide.
Pertanto, le tecniche di compilazione dovrebbero essere in grado di estrarre e sfruttare queste informazioni già presenti nelle regole del gioco per produrre fuzzer di alta qualità.

In questo lavoro proponiamo un metodo per generare fuzzer per giochi che utilizzano informazioni sulla disponibilità dinamica delle azioni.
Descriviamo una tecnica per analizzare una sequenza arbitraria di byte come una sequenza di azioni sul gioco, quindi spieghiamo come minimizzare il numero di azioni non valide nelle sequenze di azioni analizzate.
Questa tecnica è stata implementata nel compilatore del linguaggio RuleBook. RuleBook è un DSL per la descrizione dei giochi e il compilatore RuleBook ha a disposizione le informazioni rilevanti per la nostra tecnica al momento della compilazione.
Misuriamo le prestazioni di quattro versioni di fuzzer generati con questo metodo, che variano in base alla quantità di informazioni sul gioco di destinazione che utilizzano, insieme a due versioni di fuzzer generati utilizzando le implementazioni di gioco di OpenSpiel.
Utilizzando tre giochi campione, dimostriamo che i fuzzer generati dalle descrizioni RL hanno le stesse prestazioni delle loro controparti OpenSpiel.
Inoltre, dimostriamo che tra i fuzzer RL, le versioni che sfruttano maggiormente le informazioni esposte da RL sono le più performanti.
\\
\\
\textbf{Parole chiave:} compilers, fuzzing, giochi
%----------------------------------------------------------------------------
%	LIST OF CONTENTS/FIGURES/TABLES/SYMBOLS
%----------------------------------------------------------------------------

% TABLE OF CONTENTS
\thispagestyle{empty}
\tableofcontents % Table of contents 
\thispagestyle{empty}
\cleardoublepage

%-------------------------------------------------------------------------
%	THESIS MAIN TEXT
%-------------------------------------------------------------------------
% In the main text of your thesis you can write the chapters in two different ways:
%
%(1) As presented in this template you can write:
%    \chapter{Title of the chapter}
%    *body of the chapter*
%
%(2) You can write your chapter in a separated .tex file and then include it in the main file with the following command:
%    \chapter{Title of the chapter}
%    \input{chapter_file.tex}
%
% Especially for long thesis, we recommend you the second option.

\addtocontents{toc}{\vspace{2em}} % Add a gap in the Contents, for aesthetics
\mainmatter % Begin numeric (1,2,3...) page numbering

% --------------------------------------------------------------------------
% NUMBERED CHAPTERS % Regular chapters following
% --------------------------------------------------------------------------
\chapter{Introduction}
Fuzzing is a software testing technique where the program under test is fed a large number of semi-random inputs, hoping to trigger unexpected behavior.
Thanks to its demonstrated effectiveness and practicality of application, fuzzing is one of the most popular techniques to discover bugs and vulnerabilities in software.
Fuzzing was conceived by Miller \textit{et al.} in 1988 \cite{Miller1999}.
They applied the method to test the reliability of UNIX utilities, programs that accept a single input consisting of an arbitrary string.
Over the last thirty years since the conception of fuzzing, it has been utilized in a multitude of fields to discover bugs in programs with more complex interfaces, such as compilers, network protocols and operating system kernels.

Despite its popularity in a wide range of fields, fuzzing is not very practical to apply to input intensive programs, such as games.
The main reason behind this is that games often have complex input patterns.
In contrast to the UNIX utilities Miller \textit{et al.} have fuzzed, games do not operate on a single potentially very large input.
Instead, they evolve with a multitude of actions taken by the players, as well as random events.
Therefore, a fuzzer for a game needs to interpret the fuzz input, an arbitrary set of bytes, as a sequence of actions on the game.
To illustrate why this is challenging, let us inspect backgammon as an example.

In a game of backgammon, on each turn, the player has to decide whether they want to offer their opponent the doubling cube.
Then, the opponent must decide to either accept the cube or concede.
After that, the player has to roll two six-sided dice.
Finally, they have to move two pieces each by the result of one of the dice.
We can identify at least four different kinds of actions to be taken on a game of backgammon: deciding whether to offer the doubling cube, deciding whether to accept, rolling the dice, using the dice to move pieces.
The different kinds of actions are not uniform in the input they accept.
The first two accept boolean inputs, the third one accepts two integers in the range $[1,6]$, and the last one accepts a piece to move.
These actions need to be performed in a certain order.
Furthermore, this order can not be statically determined.
For example, if the rolled dice are both three, the player needs to move four pieces as opposed to two.
Finally, not all pieces are moveable by an arbitrary amount, the dice the player rolls determine whether or not a particular piece is movable for the turn.
To summarize, forming valid sequences of action to feed to games can be a very complex task.

On the other hand, invalid sequences of actions do not make interesting test cases, since the game is expected to crash with these inputs.
Therefore, fuzzers that frequently parse the fuzz input into invalid sequences of actions perform poorly.
One option to obtain fuzzers that avoid producing invalid action sequences is to write them manually.
The fuzzer developer can utilize their knowledge about the game, as well as their skill in analyzing the game logic to develop custom parsing procedures for the fuzz input.
However, this option is both costly and error prone.
We would much rather have a procedure to automatically generate such fuzzers.
Such a procedure would need to know about the actions that can be taken on the game, in which game states each of these actions can be taken, and what parameters they can be taken with.

Fortunately, the program implementing the game needs to validate the actions players take and display error messages if they are not valid.
This requires the information about the possible actions, their availability and the constraints on their parameters to be present in the target program.
Therefore, compiler techniques should be able to extract this information already present in the program to emit high-quality fuzzers that avoid generating invalid action sequences.
The relevant information is available at compile time in the compiler for RuleBook (RL), a domain specific language designed to describe games and other input intensive simulations \cite{RLC}.
Motivated by this, the contribution of this work is as follows:
\begin{enumerate}
    \item We propose a method to automatically generate fuzzers for games described in RL, integrated within the compiler.
    \item We identify two factors leading to invalid action sequences: The generation of invocations to unavailable actions and the generation of illegal arguments to available actions. Then, we describe techniques to mitigate these two factors.
    \item We generate fuzzers for games implemented in OpenSpiel to form a reference for comparison.
    \item We measure the performance of four versions of fuzzers generated by the RL compiler as well as two versions of fuzzers for OpenSpiel games, varying in how much information about the game they utilize to avoid generating invalid action sequences.
    \item We compare the performances of these fuzzers to assess the impact of utilizing this information.
\end{enumerate}

\subsection*{Thesis organization}
The remaining part of this thesis is organized as follows:
\begin{description}
    \item [Chapter 2] Explains the motivation behind RL, demonstrates the problems it aims to solve and describes the structure of the language. In particular, the language features utilized by our fuzzing technique.
    \item [Chapter 3] Provides an overview of the technologies for compilers and fuzzing utilized by our method, as well as describing fuzzing and types of fuzzers in greater detail.
    \item [Chapter 4] Details our method of automatically generating fuzzers along with the key features to improve their performance.
    \item [Chapter 5] Describes our baseline of evaluation in the form of fuzzers for OpenSpiel games, details our experimental campaign on a selection of three benchmark games and presents the experimental results.
    \item [Chapter 6] Presents our conclusions and suggests future avenues of research related to our work. 
\end{description}

\chapter{Background}
In this chapter, we introduce the motivation behind RL and provide a brief overview of the language.
Then, we describe the language features tightly linked with fuzzing in greater detail, aiming to supply the reader enough of a background to 
be able to comprehend how our fuzzer generation technique works.

\section{What is RL}
RL (RuleBook) is a domain specific language to describe multi-agent interactions, most commonly games \cite{RLC}.
RLC is the compiler for RL. RL is created to solve a problem with the way games are implemented today that makes it difficult to apply automated analysis and machine learning techniques to games.
Today, games are developed as a collection of event handlers.
The developer specifies the actions players can take, as well as the effect of those actions on the game state.
The actions are described in isolation, one has to analyze which actions can follow which others in order to simulate a game trace.
In contrast, when we describe games to other humans in natural language, as RL does, we describe them as sequential scenarios where the players get to take actions as part of the scenario.
We expect the actions to be taken at certain points in the scenario, with certain constraints.
As an example, compare two natural language descriptions of a single hand of blackjack to illustrate the difference.

A description in terms of events, their results, and their constraints could be:
\begin{itemize}
    \item Players can take additional cards.
    \item Players can pass.
    \item It is always a single player’s turn.
    \item When a player takes an additional card, they get a card from the top of the deck.
    \item A player may not take an additional card unless it is their turn.
    \item A player may not take an additional card if they have passed before.
    \item When a player takes an additional card or passes, it becomes the next player’s turn.
    \item Players start with 2 cards in their hands.
    \item If all players have passed once, the player with the hand closest to 21 wins.
\end{itemize}

On the other hand, a description as a sequential scenario could be:
\begin{itemize}
    \item The game starts by dealing each player a hand of 2 cards.
    \item Until all players have passed once, each player that has not previously passed decides whether they want to take an additional card or pass.
    \item The players with the hand closest to 21 wins.
\end{itemize}
The description organized into event handlers makes games harder to comprehend and potentially harder to develop for humans.
However, RL's main focus is not really on human readability.
It tries to solve the problem that modern games implemented as event handlers are very difficult to perform automated analyses on.
A sequential description makes game descriptions more amenable to automated analysis techniques.
These techniques can range from fuzzing the game with random actions hoping to discover invalid game states early in development, to training ML models on the game.

\section{Description of the RL language}
RL is an imperative programming language and its structure is similar to other common imperative
languages such as Python or C. An RL program is composed of functions, and the entry point is a special
function with signature \texttt{fun main() -> Int}. Each function has a list of statements to be executed sequentially.
These statements include what one might expect to find in an imperative language, such as 
variable declarations, assignments, if statements, while loops, and function calls. RL extends this 
familiar language structure with a couple key features.

\subsection{Actions}
In addition to regular functions, RL features another function-like construct called actions.
The main difference between a function and an action is in their control flow semantics.
When a function is called, it runs until it executes a return statement.
Then, all variables local to the function body are discarded and solely the return value is returned to the caller.
In contrast, when an action is invoked it runs until it executes a special kind of statement which causes it to suspend.
Then, it returns an action object that describes the current state of execution, including the values of local variables as well as where the execution has suspended.
The caller can invoke methods of this returned object to resume the execution from its previous state.
Action bodies do not contain explicit return statements, actions are meant to model processes, not values.

Two kinds of statements suspend an action's executions: Action statements and actions statements
An action statement has the syntax \texttt{act action\_name( param\_type param\_name, ...)}.
When the action suspends on an action statement, it can be resumed by calling the method of the action object that shares a signature with the action statement.
The action then resumes execution starting from the statement immediately succeeding the action statement.
In addition, the action statement's parameters are accessible by successive statements.
The parameters assume the value of arguments passed to the function that resumes the execution.

An actions statement has the following syntax
\begin{lstlisting}
actions:
    action_statement
    non_action_statement
    non_action_statement
    ...

    action_statement
    non_action_statement
    non_action_statement
    ...
\end{lstlisting}
An action suspended on an actions statement can be resumed on any of the action statements contained in it.
When resumed, only statements until the next action statement are executed before proceeding to the actions statement's successor.


Not all variables in an action's body are exposed to callers.
Variables declared with the keyword \texttt{frm} are stored in the action's frame.
They persist across suspensions and are reachable from outside the action.
On the other hand, variables declared with the keyword \texttt{let} are temporary values.
They are discarded when the action's execution suspends, just like local variables in regular functions.
As an example, consider the following program:

\lstinputlisting{action_example.rl}

The main function invokes \texttt{nim(14)}.
The action initializes the variables \texttt{current\_player} and \texttt{remaining\_sticks}, enters the loop and suspends at line 7.
When it suspends, the action returns a \texttt{Nim} object, which holds the variables
 \texttt{current\_player} and \texttt{remaining\_sticks}, exposes the functions \texttt{pick\_up\_sticks(Int count)} and \texttt{is\_done\(\)} 
 and "remembers" that the action is suspended at the \texttt{pick\_up\_sticks} action statement.
The main function then stores this object in the variable \texttt{game}, and calls the method it exposes multiple times.
Each of these calls but the last one execute one iteration of the while loop before suspending on line 7 once again.
During the last \texttt{pick\_up\_sticks} call at line 18, the action's execution exits the while loop and terminates, since there are no other action statements in the action's body.
After this point, \texttt{game.is\_done()} returns \texttt{True} and calling \texttt{pick\_up\_sticks} again will result in a crash since the action is not suspended on that statement.
The action's frame variables are still accessible through the \texttt{Nim} object.

\subsection{Preconditions}
In RL, every function, action and subaction statement can have a list of preconditions attached to it.
Any expression that evaluates to a boolean value can be a precondition, and preconditions can use the parameters of the function, action or action statement they are associated with.
Unless optimizations are turned on during compilation, a function call or action instantiation with arguments that violate its target's preconditions results in a crash.

Being able to express preconditions is crucial for describing simulations.
For instance, consider the nim example. The action 
\begin{lstlisting}
act nim(Int num_sticks) -> Nim:
    frm winner : Int
    frm current_player = 0
    frm remaining_sticks = num_sticks

    while remaining_sticks > 0:
        act pick_up_sticks(Int count)
        remaining_sticks = remaining_sticks - count
        current_player = 1 - current_player

    winner = current_player
\end{lstlisting}
allows players to pick up more sticks than there are in the game, or even a negative number of sticks.
Moreover, the game is not restricted to start with a positive number of sticks in the first place.
Enhancing the description with preconditions increases the description's readability, eases debugging and boosts the potential of automated analysis methods.
\begin{lstlisting}
act nim(Int num_sticks) {num_sticks > 0} -> Nim:
    frm winner : Int
    frm current_player = 0
    frm remaining_sticks = num_sticks

    while remaining_sticks > 0:
        act pick_up_sticks(Int count) {
            count > 0,
            count <= 4,
            count <= remaining_sticks
        }
        remaining_sticks = remaining_sticks - count
        current_player = 1 - current_player

    winner = current_player
\end{lstlisting}
To summarize, RL is a domain-specific language aimed at writing, reading and automatically analyzing game descriptions easier, as well as running machine learning methods on them.
It has a Python-like syntax. RL allows describing games as sequential scenarios mainly through actions, which are stateful suspendable procedures.
In addition, RL supports specifying preconditions for functions, actions and subactions, further facilitating the application of analysis techniques.
\chapter{State of the art}
In this chapter, we aim to provide an insight into the widely adopted technologies for compilers and fuzzing.

\section{Compiler technologies}
\subsection{LLVM}
LLVM, short for Low Level Virtual Machine, is a compiler framework designed to perform lifelong program analysis and transformation for arbitrary software. [ref here]
Lifelong analysis and transformation includes link-time techniques for interprocedural analyses, install-time techniques for machine-dependent optimizations and runtime techniques for dynamic analyses.
LLVM revolves around a common program representation independent from both the source language and the target architecture, called LLVM IR.
LLVM IR is a RISC-like instruction set enhanced with some high level information to facilitate analyses.
It includes source language independent type information, explicit control flow graphs and explicit dataflow representations using registers in Static Single Assignment form [ref here].
The design of this intermediate representation allows utilizing the LLVM framework with a wide range of source languages and target architectures, while being descriptive enough to allow powerful transformations, analyses and optimizations on the IR itself.

LLVM has a three tier architecture. First, the program in a source language is first compiled compiled into LLVM IR.
This stage includes lexical analysis, syntax analysis, semantic analysis and high-level language dependent optimizations and transformations.
The most popular example, clang, is an LLVM front-end that [ref here] compiles C, Objective C and C++ into LLVM IR.
However, clang is far from being the only adopted frontend for LLVM.
Researchers have developed LLVM frontends for to perform various tasks ranging from compiling languages like Rust [ref here] or Ruby[ref here] to creating a JIT compiler for Python [ref here].

Then, compiler language-independent optimizations, analyses and transformations are performed on LLVM IR[ref here - llvm docs].
These are implemented as independent passes on the IR.
Examples include analyses such as stack safety analysis and memory dependence analysis, as well as 
transformations such as dead code elimination, function inlining and duplicate global constant merging.
Thanks to the modular nature of this optimization and analysis stage, researchers have implemented various custom passes for tasks such as polyhedral optimization[ref here] and race detection [ref here]. 

Lastly, the optimized LLVM IR is handed over to a target architecture dependent code generator.
The code generator performs architecture-dependent optimizations, as well as tasks such as instruction selection and register allocation.
Separating this stage form the rest of the compile architecture enables developers to support a new architecture by implementing a new code generator only.

\subsection{MLIR}
MLIR, short for Multi-Level IR Compiler Framework is a framework aiming to reduce software fragmentation in compiler development, improve compilation for heterogenous hardware, facilitate the development of compilers for domain specific languages and pave the way to connect existing compiler frameworks together[ref here].
MLIR is maintained as part of the LLVM project, and is tightly integrated with LLVM.

The creators of MLIR observed that even though the LLVM framework is useful for reusing compilation techniques and algorithms that do not depend on
the source language and the target architecture, modern languages often resort to developing their own IR in order to solve domain-specific problems.
Such as library-specific optimizations or optimizing machine learning pipelines.
In addition, lowering from a decorated syntax tree for a modern language to LLVM IR is not always a straight-forward task and often necessitates one or more intermediate forms of IR, as illustrated in Figure \ref{compilationPipelines}.

Although it's certainly possible develop intermediate representations for compiler frontends, the developers end up solving many common problems such as 
location tracking, pass management and pattern based lowering.
This manifests in a decreased quality in the infrastructure of these compilers, especially for smaller domain specific languages.
MLIR solves this problem by introducing a standard for Static Single Assignment based intermediate representations.
IR's developed with MLIR can make use of the MLIR's library of solutions to common problems, and integrate with other MLIR based IR's with great ease.

\begin{figure}[h]
    \centering
    \includegraphics*[width=8cm]{Compilation_pipelines}
    \caption{Compilation pipelines of different languages. Taken from [ref here].}
    \label{compilationPipelines}
\end{figure}

\subsubsection{Structure of MLIR}
In MLIR, the unit of semantics is an "Operation", shortly referred to as Ops. Ops can represent various program concepts ranging from a single instruction or a function to a whole file or module.
An Op has one or more operands, and more or more results. Both operands and results are SSA values.
An Op's operands' definitions need to dominate the Op, and MLIR takes care of validating this constraint.
In addition, MLIR allows nesting operations via Regions. An Op may have one more more attached Regions, which can contain one or more Blocks.
Blocks are in turn sequential lists of operations.
The blocks of a region form a control flow graph.
The semantics of this control flow graph, as well as the semantics of the region itself depend on the type of Op they are attached to.
An Op may also have one or more attributes, which represent compile-time information such as constants and names.

As opposed to LLVM IR's fixed set of instructions, MLIR has an extensible set of Ops.
An MLIR based IR for a domain specific language is organized into one or more MLIR dialects.
Dialects define types of Ops, as well as value types and attributes.
The MLIR repository includes many dialects to model different types of IR [ref - MLIR docs].
For instance, there exists an LLVM dialect which models LLVM IR, a SPIR-V to represent graphics shaders and compute kernels, and a "linalg" dialect to represent linear algebra structures and operations.

During compilation, the IR does not need to consist solely of a single dialect at every instance.
The IR in general is made of operations from a collection of dialects, and it's lowered incrementally by multiple passes into the target dialect.
Most commonly, the target dialect is LLVM. But MLIR can support lowering into different dialects too.
This extensibility has been utilized to create a wide range of domain specific compilers ranging from an Open Neural Network Exchange (ONNX) compiler [ref] to a High Level Synthesis (HLS) compiler[ref].
Most notably for this thesis, RLC itself is implemented as an MLIR dialect and a collection of passes to lower it to LLVM IR.

\section{Fuzzing}
Fuzzing, as introduced by Miller \textit{et al.} in 1998[ref here], is an automated software testing technique where the program under test is invoked with randomly generated test cases hoping to trigger bugs.
A fuzzer is a program that generates the test cases. Although Miller \textit{et al.}'s idea was as simple as invoking UNIX utilities with random strings, fuzzers today are much more sophisticated.
As an example, American Fuzzy Lop (AFL) [ref here] applies some lightweight instrumentation to the branch instructions of the compiled programs it tests.
Then, it measures how many new CFG edges are traversed by each fuzz input at runtime.
It generates new fuzz inputs by mutating the ones that result in high coverage.
On the other hand, SlowFuzz [reference] tracks the resource usage of its test target to find inputs that trigger algorithmic complexity vulnerabilities.
Today, fuzzing is an increasingly popular technique to find software vulnerabilities.

\subsection{White-box vs. black-box fuzzers}
Fuzzers can be categorized in terms of how much information they require about the program under test.
Black-box fuzzers require no information about their target.
They either generate inputs completely randomly or mutate an initial library of well-known inputs using a set of pre-defined rules.
On the other hand, white-box fuzzers know everything there is to know about the program under test.
One example, DART [ref], first feeds random inputs to the program under test, symbolically executes the discovered traces to gather a set of path constraints on them, then uses a solver to generate inputs that are guaranteed to discover new paths.
In practice, most widely used fuzzers today fall into a category in between these two, gray-box fuzzing.
They make use of partial information about the program under test.
This information is generally obtained by instrumenting the program.

For the fuzzers described as part of our work, we use the term black-box for the fuzzers that make the smallest use of the information available about the fuzzed games.
Similarly, we will use the term white-box to describe the fuzzers that use most of the available information, compared to the alternatives described in this thesis.
In reality, all fuzzers implemented as part of this thesis can be described as grey-box fuzzers.
Nevertheless, we find the terms black-box and white-box useful to highlight the difference in the fuzzers we implement.

\subsection{libFuzzer - Fuzzing in LLVM}
LibFuzzer [ref] is the fuzzing engine integrated with the LLVM project.
Being integrated with the Clang compiler, libFuzzer is directly linked against the program under test, rather than running as a separate process that invokes the program under test.
The link between libFuzzer and the program under test is formed by a function called the fuzz target.
The fuzz target is a C function that should call the API of the program under test utilizing the fuzz input. 
It has the following signature:

\begin{lstlisting}  
int LLVMFuzzerTestOneInput(const uint8_t *Data, size_t Size) {
    DoSomethingInterestingWithMyAPI(Data, Size);
    return 0;
}
\end{lstlisting}

Since it is integrated with the LLVM framework, libFuzzer can make use of various LLVM sanitizers.
For instance, information provided by Address Sanitizer (ASAN) [ref] enables libFuzzer to detect bugs that might otherwise happen silently, such as buffer overflows or memory leaks.

Futhermore, being a gray-box fuzzing engine, libFuzzer uses instrumentation on the program under test to track the basic blocks visited by each fuzz input.
This instrumentation is provided by LLVM's SanitizerCoverage [ref].
The instrumentation being external to libFuzzer and included in LLVM is vital to the method we introduce in this thesis.
Even though, we do not use the whole Clang pipeline, we can integrate SanitizerCoverage after lowering RL to LLVM IR.
This enables us to seamlessly provide the coverage information to libFuzzer with little effort.
\chapter{Solution design}\label{solutionDesign}

As previously stated, our goal is to implement a method of automatically generating
 efficient fuzzers for actions described in RL.
In this chapter, we describe how we accomplish that goal.
We start by describing a simple method to generate black-box fuzzers that uses almost none of the information available in an RL action description.
Then, we build on this method incrementally, integrating parts of the available information one by one in order to improve performance.

\section{Conceptual Design}
Within the scope of this thesis, we only consider fuzzers integrated with LLVM's libFuzzer.
LibFuzzer interfaces with the fuzzed action through a fuzzing entrypoint called the fuzz target.
The fuzz target is a function that accepts an array of bytes and does something interesting with these bytes using the API under test \cite{LibFuzzer}.
LibFuzzer invokes this function repeatedly with different fuzz inputs. It tracks which areas of the code are reached, and generates mutations on the corpus of input data in order to maximize the code coverage.
Utilizing libFuzzer, we narrow the task of generating a fuzzer down to the task to generating a fuzz target.
The fuzz target should use the fuzz input it receives to interact with the interface of the fuzzed action.

For the methods described in this section, we model the fuzz target as a function written in pseudo-code that has access to the fuzz input as well
 as the public methods of the fuzzed action.
In reality, the fuzz target needs to be written in C and the action's interface is written in RL. We describe how the two are connected in Section \ref{architecture}.

Throughout this section, we will use the following action description to exemplify the methods we discuss:
\begin{lstlisting}
act nim() -> Nim:
    frm winner : Int
    frm current_player = 0
    frm remaining_sticks

    act decide_num_sticks(Int num_sticks) {num_sticks > 0}
    remaining_sticks = num_sticks

    while remaining_sticks > 0:
        act pick_up_sticks(Int count) {
            count > 0,
            count <= 4,
            count <= remaining_sticks
        }
        remaining_sticks = remaining_sticks - count
        current_player = 1 - current_player

    winner = current_player
\end{lstlisting}
The action is a slightly modified version of the Nim example from the previous section.
In this version, the initial number of sticks is picked through an action. This change makes the examples more illustrative by introducing more than one action.

\subsection{Generating black-box fuzz targets} \label{blackboxFuzzTargets}
As a baseline for our discussion, let us describe a simple method to generate black-box fuzz targets for RL actions.
A black-box fuzz target knows about the subactions available in the action's interface, and their signatures.
The simplest black-box fuzz target uses some portion of the fuzz input to decide which subaction to call.
Then, it generates arguments for each of the picked subaction's parameters.
Finally, it calls the subaction with the generated arguments.
This method interprets the fuzz input as an action call, and tests whether that action call results in problematic behavior.

A fuzz target for the Nim example would look like the following:
\begin{algorithm}[H]
    \caption{Black-box fuzz target for Nim}
    \begin{algorithmic}[1]
    \STATE $game \gets nim()$
    \STATE $actionIndex \gets pickValue(fuzzInput, 0, 1)$
    \IF{$actionIndex = 0$}
        \STATE $arg0 \gets pickValue(fuzzInput, INT\_MIN, INT\_MAX)$
        \STATE $game.decide\_num\_sticks(arg0)$
    \ENDIF
    \IF{$actionIndex = 1$}
        \STATE $arg0 \gets pickValue(fuzzInput, INT\_MIN, INT\_MAX)$
        \STATE $game.pick\_up\_sticks(arg0)$
    \ENDIF
    \end{algorithmic}
\end{algorithm}
Where \texttt{pickValue(byte[] fuzzInput, int min, int max)} is a function that consumes the next $log_2(max - min + 1)$ bits of \texttt{fuzzInput}
 to produce an integer between \texttt{min} and \texttt{max}.
 \texttt{INT\_MIN}, \texttt{INT\_MAX} represent the bounds of an integer value.

\subsection{Performing a sequence of actions}
One limitation of this simple black-box fuzzer is that it never executes more than one subaction of the fuzzed action.
Therefore, it will not be able to discover bugs that occur only after multiple action calls.
To amend this shortcoming, we can make the fuzz target repeat this process until it consumes every bit of the fuzz input.
This method interprets the fuzz input as a sequence of action calls, instead of a single action call.
In this way, the fuzzer will be able to capture the stateful behavior of RL actions.

Applied to the Nim example, this method would produce the following fuzz target.
\begin{algorithm}[H]
    \caption{Fuzz target performing multiple actions for Nim}
    \begin{algorithmic}[1]
    \STATE $game \gets nim()$
    \WHILE {fuzz input is long enough}
        \STATE $actionIndex \gets pickValue(fuzzInput, 0, 1)$
        \IF{$actionIndex = 0$}
            \STATE $arg0 \gets pickValue(fuzzInput, INT\_MIN, INT\_MAX)$
            \STATE $game.decide\_num\_sticks(arg0)$
        \ENDIF
        \IF{$actionIndex = 1$}
            \STATE $arg0 \gets pickValue(fuzzInput, INT\_MIN, INT\_MAX)$
            \STATE $game.pick\_up\_sticks(arg0)$
        \ENDIF
    \ENDWHILE
    \end{algorithmic}
\end{algorithm}

\subsection{Avoiding expected crashes}\label{avoidingCrashes}
Another critical shortcoming of this method is that invoking an arbitrary subaction with arbitrary parameters may result in an expected crash.
For example, calling \texttt{pick\_up\_sticks} before calling \texttt{decide\_num\_sticks} is expected to cause a crash since the action will not have paused on the correct subaction.
In addition, calling \texttt{decide\_num\_sticks(-1)} will also result in a crash since the precondition of the subaction is violated.
If a call is expect to cause a crash, we describe the call as illegal.

When we make an illegal call, the program will crash and the fuzzer will report a bug.
However, we are not interested in these crashes since they are a result of how the fuzz target interacts with the action, not a result of how the action is described.
To solve this problem, we need some runtime mechanism to decide the legality of a call.
We assume such a mechanism to be available for now and describe its implementation in Section \ref{decidingLegality}.
Then, the fuzz target we generate should check the legality of all subaction calls it makes, and avoid making illegal calls.
When it generates an illegal call, the fuzz target can simply continue to the next iteration of the main loop and generate a new call.
This method interprets the fuzz input as a sequence of legal action calls.

For the Nim example, we would obtain the following fuzz target:
\begin{algorithm}[H]
    \caption{Fuzz target performing multiple actions for Nim}
    \begin{algorithmic}[1]
    \STATE $game \gets nim()$
    \WHILE {fuzz input is long enough}
        \STATE $actionIndex \gets pickValue(fuzzInput, 0, 1)$
        \IF{$actionIndex = 0$}
            \STATE $arg0 \gets pickValue(fuzzInput, INT\_MIN, INT\_MAX)$
            \IF {$game.decide\_num\_sticks(arg0)$ is a legal call}
                \STATE $game.decide\_num\_sticks(arg0)$
            \ENDIF
        \ENDIF
        \IF{$actionIndex = 1$}
            \STATE $arg0 \gets pickValue(fuzzInput, INT\_MIN, INT\_MAX)$
            \IF {$game.pick\_up\_sticks(arg0)$ is a legal call}
                \STATE $game.pick\_up\_sticks(arg0)$
            \ENDIF
        \ENDIF
    \ENDWHILE
    \end{algorithmic}
\end{algorithm}

With these improvements, we have a functionally correct design.
However, Even if we never make illegal calls, wasting fuzz input bits by generating illegal calls decreases the fuzzer's performance.
We can reduce the number of illegal calls we generate, and therefore improve the efficiency of the fuzzers by making use of more information available in the action description.

\subsection{Filtering out unavailable subactions}\label{filteringUnavailableSubactions}
The first observation we make about the legality of subaction calls is that any subaction is always illegal if the action has not paused on the \texttt{ActionStatement} for that subaction, or an \texttt{ActionsStatement} containing that \texttt{ActionStatement}.
With this observation, we can dynamically classify subactions as "available" or "unavailable".
If we can guarantee we never generate a call to an unavailable subaction, we eliminate a large portion of generated illegal calls.
To achieve that, we need a mechanism to dynamically decide whether a subaction is available.
Assuming we can implement this mechanism, we can make sure to only pick available subactions.
Our implementation of this mechanism is described in Section \ref{decidingAvailability}.

Here is an example for the Nim game:
\begin{algorithm}[H]
    \caption{Fuzz target performing multiple actions for Nim}
    \begin{algorithmic}[1]
    \STATE $game \gets nim()$
    \WHILE {fuzz input is long enough}
        \STATE $availableSubactions \gets []$
        \IF {$decide\_num\_sticks$ is available}
            \STATE $availableSubactions.push(0)$
        \ENDIF
        \IF {$pick\_up\_sticks$ is available}
            \STATE $availableSubactions.push(1)$
        \ENDIF
        \STATE $pickedIndex \gets pickValue(fuzzInput, 0, len(availableSubactions))$
        \STATE $actionIndex \gets availableSubactions[pickedIndex]$
        \IF{$actionIndex = 0$}
            \STATE $arg0 \gets pickValue(fuzzInput, INT\_MIN, INT\_MAX)$
            \IF {$game.decide\_num\_sticks(arg0)$ is a legal call}
                \STATE $game.decide\_num\_sticks(arg0)$
            \ENDIF
        \ENDIF
        \IF{$actionIndex = 1$}
            \STATE $arg0 \gets pickValue(fuzzInput, INT\_MIN, INT\_MAX)$
            \IF {$game.pick\_up\_sticks(arg0)$ is a legal call}
                \STATE $game.pick\_up\_sticks(arg0)$
            \ENDIF
        \ENDIF
    \ENDWHILE
    \end{algorithmic}
\end{algorithm}

\subsection{Utilizing preconditions}
Now that we made sure we never pick unavailable subactions, the only remaining cause of generated illegal subaction calls is violated preconditions.
Preconditions in RL can be arbitrarily complex. They can include any kind of expression RL supports, including calls to arbitrary functions.
Therefore, given a set of conditions, finding a set of arguments that satisfy them is not an easy task.
As a result, we can not guarantee to never pick illegal arguments to a subaction call.
Nevertheless, we can focus on particular forms of preconditions and extract some information from them to decrease the number of illegal arguments we generate.

As an example, consider the subaction arguments in the Nim action.
We can replace \texttt{pickValue(fuzzInput, INT\_MIN, INT\_MAX)} at line 13 with \texttt{pickValue(fuzzInput, 1, INT\_MAX)}, since we know all negative choices are going to be discarded.
Furthermore, when picking the argument of \texttt{pick\_up\_sticks}, we can \texttt{pickValue(fuzzInput, 1, min(4, game.remaining\_sticks))}, so that we guarantee never picking an argument out of bounds.

The details of constraint types we consider in the scope of this thesis are described in Section \ref{preconditionAnalysis}. 

\section{Implementation}
In this section, we explain how we implement the method described in the previous section.
We provide an architectural overview, then we detail how we address the challenges we have highlighted.

\subsection{Architectural overview} \label{architecture}
As previously explained, a fuzzer built with libFuzzer needs to expose a fuzz target.
The fuzz target is a C function that accepts the fuzz input and calls the fuzzed action's functions in some sequence, looking for unexpected behavior.
The key challenge in generating fuzz targets is that they need to have access to both the fuzz input, which is a C byte array; and the action's subaction functions, which are available in RL.
This requires part of the implementation to be in C or C++, and part of it to be integrated with RLC.
The generated fuzz target needs to be customized to the number of subactions the action has, their signatures, and their preconditions.

One option is implement the body of the fuzz target in C or C++.
In this case, we need to express the information related to the fuzzed action as either macros or templates.
Then, we can extend RLC to emit a C header containing the relevant information.
Finally, the compiler can perform the necessary macro expansions and template instantiations to arrive at the customized fuzzer.
We have explored this option extensively and generated fuzzers covering a subset of the features described in the previous section.
However, we ultimately found it very difficult to read and maintain the fuzz target body written in terms of these macros.
In addition, designing macro representations for RL concepts to be serializable into a C header without losing descriptive power proved to be challenging.
Therefore, we ultimately decided against this option.

Instead, we extend RLC to emit a global function implementing the body of the fuzz target when it's passed the \texttt{--fuzzer} flag.
This global function has a static signature, independent of the characteristics of the action to be fuzzed.
The fuzz target invoked by libFuzzer simply calls this function and then returns.
Since RLC knows about all characteristics of the action to be fuzzed at compile time, it can customize this function accordingly, without the need of designing serializable representations for these characteristics.

On the other hand, it's difficult for RLC to emit the functions that interpret sections of the fuzz input as different data types.
For instance, the \texttt{pickValue(fuzzInput, min, max)} function utilized in the previous section should parse the next $log_2(max - min + 1)$ bits of 
    the fuzz input as an integer in the range $[min, max]$.
A proper implementation of this function requires pointer arithmetic and bitwise operators.
Fortunately, these functions do not depend on the characteristics of the action to be fuzzed.
Hence, we implement these in C++ and the code emitted by RLC issues calls to them wherever necessary.

Furthermore, we extend RLC with a fuzzer standard library.
This library contains functions, written in RL, that implement parts of the fuzz target body which don't depend on the fuzzed action's characteristics.
For instance, the logic that initializes and maintains a vector of available subaction indices is implemented in RL source code.
This increases the readability and maintainability of the RLC extension that emits the fuzz target.
Since compiling RL source code is not a challenge for RLC, and emitting new operations that don't derive from a source code rapidly becomes very difficult to maintain.

To sum up, the fuzz target is generated in three distinct components that are then all linked together.
The fuzz target called by libFuzzer and the utility functions to parse the fuzz input are written in C++ and compiled by a C++ compiler while building RLC.
The parts of the actual fuzz target body that do not depend on the fuzzed action's characteristics are written in RL, and they are shipped with RLC as part of the standard library.
The parts of the actual fuzz target that depend on the fuzzed action's characteristics are instead emitted by RLC at compile time if the \texttt{--fuzzer} option is specified.
Linking these three components, as well as libFuzzer itself, produces the final fuzzer binary.

\subsection{Deducing the availability of subactions}\label{decidingAvailability}
Available subactions are subactions which the action can resume from. A subaction is available if the execution of the action has last suspended on the \texttt{ActionStatement} corresponding to that subaction or an \texttt{ActionsStatement} containing it.
In Section \ref{filteringUnavailableSubactions}, we described how we can generate subaction calls only to available subactions.
As a prerequisite, we assumed the availability of some mechanism to decide whether an action is available.
In this section, we explain the relevant implementation details of RLC and describe how we implemented this mechanism.

To start with, we should describe how subaction functions called at the wrong time cause crash.
Recall that action calls return action objects that keep track of the state of the action.
This state includes frame variables, as well as where the action has last suspended.
The first field of the returned action object is always \texttt{resumeIndex}.
\texttt{resumeIndex = 0} means the action has not started executing yet, and will start from the beginning of its body.
When suspending on an \texttt{ActionStatement} or \texttt{ActionsStatement}, a \texttt{resumeIndex} corresponding to that statement is stored in the action object.
The subaction functions all accept this action object as their first argument and they implicitly have a precondition checking whether the \texttt{resumeIndex} stored in it is the expected one.
When a subaction function is called at the wrong time, this precondition fails, resulting in a crash.

Taking this into account, it's sufficient to check whether the stored \texttt{resumeIndex} is the one expected by the subaction function to dynamically decide its availability.

\subsection{Deducing the legality of subaction calls}\label{decidingLegality}
As alluded to in Section \ref{avoidingCrashes}, we need a mechanism to dynamically decide whether a subaction call with a particular set of arguments is expected to result in a crash.
In this section, we explain how this mechanism is implemented.

We make two extensions to RLC in order to be able to check the legality of subaction arguments.
First, we introduce an additional IR node, called \texttt{CanOp}.
\texttt{CanOp}s have a single operand and a single result, both of type \texttt{Function}.
A \texttt{CanOp} returns a function that accepts the same arguments as its operand, and has a boolean return type.
The function returns whether its operand's precondition evaluates to true with the arguments passed to it.
Second, we introduce a new compiler pass.
This pass walks through each function in the module, emits a global function that checks its precondition, then replaces the results of all \texttt{CanOp}s referring to the original function with the newly emitted precondition checker.

Having introduced \texttt{CanOp}, the emitted fuzz target can deduce whether a subaction call is legal simply by emitting a \texttt{CanOp} to the subaction function, and a call to the \texttt{CanOp}'s result with the picked arguments.
This allows the emitted fuzzer to avoid running into expected crashes.

\subsection{Precondition Analysis} \label{preconditionAnalysis}
The last major improvement to the fuzzers we generate is utilizing RL preconditions to avoid picking subaction arguments for which the subaction's precondition evaluates to false.
Before we detail our implementation, we should discuss the difficulty of this problem in the general case.
The precondition of an RL subaction consists of a set of boolean expressions implicitly in conjunction.
These boolean expressions are not limited in the scopes and types of sub-expressions they can include.
They may include function calls, and the function calls might be blocking, they may have side effects and they may be non-deterministic.
Even if we could assume that preconditions consist only of pure, non-blocking, deterministic expressions, finding a set of arguments to satisfy their conjunctions is no easier than answering arbitrary SMT queries.

Given this perspective, our approach is to handle some subset of preconditions and extract the maximum information from those.
We can not claim to always be able to generate legal arguments, but we can handle some of the most frequent types of constraints to greatly reduce the number of illegal arguments we generate.
Our technique is a best-effort, not a definitive solution.

We restrict our precondition analysis to integer arguments to subaction functions, as well as struct arguments that are composed solely of integers and other such structs.
For each integer argument, or integer field of a struct argument of a subaction call, we dynamically decide a minimum and a maximum value.
Such that any value greater than the maximum or smaller than the minimum certainly invalidates the precondition.
Then, we pick the argument to be any integer in this range using the $pickValue(fuzzInput, min, max)$ function implemented in C++ introduced previously.
Although this is far from a perfect model of arbitrary constraints on integers, we have found it to be descriptive enough to significantly boost fuzzing performance.

Let us now describe how we decide the minimum and maximum values dynamically.
For simplicity, we focus only on integer arguments here.
We start by defining a classification on expressions that comprise the precondition in terms of the availability of their evaluation at runtime.
We classify expressions into two categories:
\begin{description}
    \item [Bound expressions] are expressions such that we can decide the value they will evaluate to while evaluating the precondition before we evaluate the precondition.
These include constants, local variables, global variables and combinations of these with deterministic operations, as well as deterministic function calls.
    \item [Unbound expressions] are expressions such that it's not possible to know the value they will evaluate to before evaluating the precondition.
    These include arguments of the subaction, nondeterministic function calls and any expression that them as sub-expressions.
\end{description}

Note that an argument of the subaction does not have to be an unbound value.
In fact, since RLC implements action objects' member functions as global functions with the implicit first argument pointing to the object, the first argument of any subaction function call is always a bound expression, it evaluates to the action object.
Having introduced this terminology, our goal is to find the minimum and maximum values the unbound arguments of a subaction function can take while satisfying the function's precondition.

We introduce a new compiler pass to emit the code that decides the minimum and maximum values dynamically.
We start by normalizing the precondition to be a disjunction of conjunction of terms, where each term is a value produced by something other than an \texttt{AndOp} or an \texttt{OrOp}.
In other words, we express the precondition in the form $(t_1 \land t_2 \land ...) \lor (t_3 \land t_4 \land ...) \lor ...$.
Then, we emit a block of code for each unbound argument that computes its minimum and maximum values.

For each conjunction $(t_1 \land t_2 \land ...)$ in the normalized precondition, we classify its terms depending on how they relate to \texttt{arg}.
\begin{description}
    \item [Conditions] are terms that consist solely on bound expressions. A condition itself is bound and can be evaluated without invoking the subaction precondition.
    \item [Constraints] are terms that include \texttt{arg} as a sub-expression, and no other unbound expressions.
    These terms impose constraints on \texttt{arg} that can be entirely expressed before the invocation of the precondition.
\end{description}
The rest of the terms depend on one or more unbound expressions other than \texttt{arg}.
For this analysis, we assume they do not constrain \texttt{arg} in any capacity.
This assumption does not always hold, and that might result in loose minimum and maximum bounds, but we simply can not decide the constraints imposed by these terms on \texttt{arg} before evaluating the precondition.
Discarding these terms is as far as our best-effort approach can stretch.

At runtime, before evaluating the precondition, we can evaluate whether all conditions of a conjunction $(t_1 \land t_2 \land ...)$ hold.
If all conditions hold, the constraints of these conjunctions are said to be active.
Assuming we can decide the minimum and maximum values imposed on \texttt{arg} by a single constraint, we can compute the aggregate minimum and maximum values by intersecting the ranges imposed by the terms of each active conjunction,
then computing the union of ranges imposed by all active conjunctions.
We can achieve this by emitting an if statement of the following form for each conjunction:

\begin{algorithm}[H]
    \caption{Aggregating the minimum and maximum values imposed by individual constraints}
    \begin{algorithmic}[1]
    \STATE $aggregate\_min \gets \infty$
    \FOR {conjunction $(cond_1 \land cond_2 \land constr_1 \land constr_2) in normalized precondition$}
        \IF {$cond_1 \land cond_2$}
            \STATE $current\_min \gets -\infty$
            \IF {$imposed\_min(constr_1) > current\_min $}
                \STATE $current\_min \gets imposed\_min(constr_1)$
            \ENDIF
            \IF {$imposed\_min(constr_2) > current\_min $}
                \STATE $current\_min \gets imposed\_min(constr_2)$
            \ENDIF
        \ENDIF
        \IF {$current\_min <  aggregate\_min$}
            \STATE $aggregate\_min \gets current\_min$
        \ENDIF
    \ENDFOR
    \end{algorithmic}
\end{algorithm}
and similarly for the maximum value.

As for how how we decide the minimum and maximum values imposed by individual constraints, we adopt a similar best-effort approach where we handle some forms of constraints that can easily be handled.
For more complex constraints, we simply assume they constrain the \texttt{arg} to be in the range $(-\infty, \infty)$.

We handle constraints that are binary operations where any one of the two operands is \texttt{arg} and the operation is one of $<$, $>$, $\le$, $\ge$, $=$.
In addition, we handle the constraints which are \texttt{CallOp}s with the callee being the result of a \texttt{CanOp}.
In this case, we apply a recursive constraint analysis to the underlying callee, mapping the unbound expressions in the current analysis context to the new analysis context to decide how \texttt{arg} is constrained by the precondition of the callee.
Recursively analyzing preconditions of \texttt{CallOp}s in the precondition is critical to the applicability of this technique because RLC has some passes that wrap an action function inside of another action function.
We would not be able to utilize the preconditions of wrapped action functions without handling these constraints.
\chapter{Evaluation}
\section{Goals}

\section{Conditions}

\section{Baseline}
In this section, we describe the baseline we use to evaluate the efficiency of fuzzers generated by RLC.
As a baseline, we need a simpler method of automatically generating fuzzers for game descriptions. Any such method
needs to establish an abstraction for how a game is described. We have chosen to use the abstraction of OpenSpiel, 
Google DeepMind's framework for applying reinforcement learning methods to games (TODO: I probably need to describe
 OpenSpiel in greater detail.).

\section{Results}
\chapter{Conclusion}
In this work, we have provided an overview of RL, a novel language conceived to describe games as sequential scenarios as opposed to event handlers.
In particular, we have highlighted the language features that boost the potential of automated analysis techniques: actions, subactions and preconditions.
We have described a method of automatically generating fuzzers using the information available through these language features, and detailed how this method is implemented as part of the RL compiler.

In addition, we have identified two major factors diminishing the performance of game fuzzers: Generating invocations to game actions which are not available in the current game state and 
    choosing invalid arguments to available game actions.
We have described a technique to completely avoid generating invocations to unavailable actions.
Regarding invalid action arguments, we have argued that it is not feasible to design a method that picks arguments satisfying arbitrary constraints in reasonable time.
Instead, we have introduced a method of analyzing some forms of constraints to mitigate the negative impact of this factor, even if it is limited in extent.

As a reference for comparison with automatically generated RL fuzzers, we have generated a black-box fuzzer and white-box fuzzer for a selection of three games implemented in OpenSpiel.
We have artificially introduced bugs in these games for the fuzzers to find.
In addition, we have implemented the same games in RL and generated four versions of fuzzers for them, differing in whether they include the two key performance improvements.
Measuring the performance of these fuzzers in terms of average fuzzing time and average number of fuzz inputs tested before finding the bug, we were able to demonstrate that the fuzzers generated from the RL descriptions performed better than the black-box OpenSpiel fuzzer.
They also outperform the white-box OpenSpiel fuzzer for the one benchmark where our constraint analysis technique can model the constraints perfectly, but fall shorter when the constraints are more complex.
We have also demonstrated that our two key improvements, utilizing the information available in the RL description, result in a significant performance increase in the generated fuzzers.

Our results suggest that future work in this topic should focus on improving the constraint analysis.
We have obtained the best results on the benchmark where our technique can describe the constraints on subaction arguments well, and observed the performance plunge when the contrary is true.
In addition, our technique may be reproduced using a fuzzing engine other than libFuzzer and the results can be compared to discover whether our findings are generalizable across different fuzz input generation methods.



%-------------------------------------------------------------------------
%	BIBLIOGRAPHY
%-------------------------------------------------------------------------

\addtocontents{toc}{\vspace{2em}} % Add a gap in the Contents, for aesthetics
\bibliography{Thesis_bibliography} % The references information are stored in the file named "Thesis_bibliography.bib"

%-------------------------------------------------------------------------
%	APPENDICES
%-------------------------------------------------------------------------

\cleardoublepage
\addtocontents{toc}{\vspace{2em}} % Add a gap in the Contents, for aesthetics
% \appendix
% \chapter{Appendix A}
% If you need to include an appendix to support the research in your thesis, you can place it at the end of the manuscript.
% An appendix contains supplementary material (figures, tables, data, codes, mathematical proofs, surveys, \dots)
% which supplement the main results contained in the previous chapters.

% \chapter{Appendix B}
% It may be necessary to include another appendix to better organize the presentation of supplementary material.

% LIST OF FIGURES
\listoffigures

% LIST OF TABLES
% \listoftables

% % LIST OF SYMBOLS
% % Write out the List of Symbols in this page
% \chapter*{List of Symbols} % You have to include a chapter for your list of symbols (
% \begin{table}[H]
%     \centering
%     \begin{tabular}{lll}
%         \textbf{Variable} & \textbf{Description} & \textbf{SI unit} \\\hline\\[-9px]
%         $\bm{u}$ & solid displacement & m \\[2px]
%         $\bm{u}_f$ & fluid displacement & m \\[2px]
%     \end{tabular}
% \end{table}

\cleardoublepage

\end{document}
